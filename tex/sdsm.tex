\documentclass[runningheads]{llncs}

\usepackage[ngerman]{babel}
\usepackage{graphicx}
%Darstellung deutscher Sonderzeichen (Umlaute etc)
\usepackage[utf8]{inputenc}
\usepackage[T1]{fontenc}
%Darstellung von URLs
\usepackage{url}
\usepackage{amsmath,amssymb} 
\usepackage{listings}
\usepackage{subfig}

\begin{document}

\section{Einleitung}

Schatten liefern wichtige Informationen über die Relation zwischen Objekten in der Computergrapfik.
Shadow Maps sind eine weit verbreitete Technik um Schatten zu rendern.



\section{Related Work}

Parallel Split Shadow Maps ist eine Technik zur Generierung von Z-Partitionen, bei der die Tiefe linear oder logarithmisch zwichen Near- und Far-Plane des View-Frustums interpoliert wird. Diese Methode benötigt einen zusätzlichen Parameter um festzulegen, wie zwischen linearer und logarithmischer Tiefe linearkombiniert werden soll, da eine rein lineare Interpolation meistens die nahen Z-Partitionen zu groß wählt, und eine rein logarithmische Interpolation diese zu klein werden lässt.


\section{Algorithmus}

Sample Distribution Shadow Maps ist eine Technik, die den Depthbuffer benutzt, um sinnvolle Z-Partitionen zu generieren.


Zunächst werden zwei Reduktion auf dem Depthbuffer durchgeführt um die Near- und Far-Plane des Kamerafrustums dicht an die sichtbare Szenengeometrie zu schieben.



\section{Implementierung}




\end{document}
