\documentclass[runningheaders,a4paper]{llncs}

\usepackage[ngerman]{babel}
\usepackage{graphicx}
\usepackage[utf8]{inputenc}
\usepackage[T1]{fontenc}
% Darstellung von URLs
\usepackage{url}
\usepackage{amsmath,amssymb} 
\usepackage{listings}
\usepackage{subfig}
\usepackage{multirow}
\usepackage{array}

% Benötigte Angaben für die Titelseite
\title{ Moment Shadow Mapping, Moment Based Volumetric Obscurance und Sample Distribution Shadow Maps }

\author{
	Akasha Jethwa 
	Matrikel-Nr: 2563201
	\and
	Jonas Weinz 
	Matrikel-Nr: 2571421
	\and
	Tom Kneiphof 
	Matrikel-Nr: 2662506
}

% Hier das Institut angeben
\institute{Institut für Informatik der Universität Bonn}

\begin{document}

% Erstellung des Titels
\maketitle

\begin{abstract}
Dieser Bericht ist Teil einer Projektgruppe am Institut für Informatik der Universität Bonn,
die sich mit der Implementierung drei verschiedener Echtzeit-Renderingtechniken für Schatten 
beschäftigt. In diesem Bericht wird die Funktionsweise und Implementierung von Sample Distribution Shadow Maps, 
Volumetric-Obscurance und Moment Shadow Mapping. Alle drei Techniken werden in einer eigenen Engine implementiert und
durch ein Demo-Spiel dargestellt.
\end{abstract}

\section{Einleitung}
Rendern von Schatten ist ein wichtiger Aspekt der interaktiven 3D Grafik. Die Struktur einer Szene, wirkt erst durch Schatten realisitisch und hilft dem Betrachter dabei, diese zu verstehen.
Allerdings ist das Berechnen der verschiedenen Schattentechniken auch aufwendig und oft anfällig für Fehlerartefakte. Deswegen untersuchen wir in diesem Paper 3 Techniken, welche verschiedene Lösungsansätze und Vorteile bieten.
\subsection{Sample Distribution Shadow Maps}
Hier könnte ein Text stehen
\subsection{Volumetric Obscurance}
Hier könnte ein Text stehen
\subsection{Moment Shadow Mapping}
Wir wollen eine Schattentechnik verwenden, die möglichst hohe Qualität liefert und dabei performant bleibt. Moment Shadow Mapping liefert uns dieses Verfahren und wird im weiteren Verlauf erläutert und implementiert.

\section{Related Work}

TODO subsections or related work embeddet into techniques

Moment Shadow Mapping \cite{msm}, Variance Shadow Maps \cite{donnelly2006variance}, Percentage Closer Filtering \cite{reeves1987rendering}

Volumetric Obscurance \cite{loos2010volumetric}

Sample Distribution Shadow Maps \cite{lauritzen2011sample}

\section{Sample Distribution Shadow Maps}

Sample Distribution Shadow Maps (SDSMs) ist ein Verfahren zur adaptiven Optimierung der Shadow Map Auflösung unter Verwendung des Camera-Space Depth Buffers.

\subsection{Algorithmus}

% Cascaded Shadow Maps aka Z-Partitioning
SDSMs basieren auf Z-Partitoning, wobei das Kamera Frustum entlang der Blickrichtung in mehrere Teilfrusta partitioniert wird. Für jedes Teilfrustum wird eine eigene Shadow Map generiert.
Auf diese Weise kann mehr Shadow Map Auflösung für nahe Objekte verwendet werden, und weniger Auflösung für weit entfernte Objekte.

% Partitionierung
Für gute Ergebnisse ist eine geeignete Wahl der Partition ausschlaggebend.
Anstatt auf eine statische Partitionierung des Kamera Frustums zu setzen wird die Partitionierung von SDSMs in jedem Frame neu berechnet.

% - Min/Max Reduktion
% - Logarithmische Partitionierung
Zunächst wird die tatsächliche Near und Far Distanz der sichtbaren Geometrie im Kamera Frustums ermittelt. Dies geschieht mittels Min/Max Reduktion der Tiefenwerte im Kamera Depth Buffer.
Anschließend wird das Kamera Frustum zwischen tatsächlicher Near und Far Distanz logarithmisch partitioniert um das perspektivische Aliasing zu minimieren.

% Abgrenzung PSSM Logarithmic/Linear
PSSMs kennen die tatsächliche Near und Far Distanz nicht, und müssen mit der Near und Far Distanz der Projektionsmatrix arbeiten.
Eine logarithmische Partitionierung würde zu viel Shadow Map Auflösung in den sehr nahen Bereich verschieben, der oft leer ist.
PSSMs konvexkombinieren die Partitionsgrenzen der linearen und logarithmischen Partitionierung um mehr Auflösung von der Near Plane weg zu bewegen.
Optimale Koeffizienten müssen meist von Hand ermittelt werden.

% (K-Means and friends)
Weitere Möglichkeiten das Kamera Frustum intelligent zu partitionieren sind ein adaptiv logarithmisches und ein K-Means Verfahren.
Das adaptiv logarithmische Verfahren vermeidet Lücken im Histogramm des Kamera Depth Buffers, indem es die Partitionsgrenzen an das Ende von diesen schiebt.
Das K-Means Verfahren sucht zunächst Maxima im Histogramm des Kamera Depth Buffers und legt dann die Partition entsprechend um die gefundenen Maxima.
Für beide Verfahren muss zunächst ein Histogramm berechnet werden.


% Betrachte Punkte in Light-Space TODO
Der Algorithmus betrachtet die Verteilung aller sichtbaren Punkte der Szene aus der Sicht des Lichtes.
Hierzu wird für jedes Sample aus dem Kamera Depth Buffer die Position in der Welt rekonstruiert und in den Light Space transformiert.


% Tight Partition Frusta

% Andere Algorithmen benutzen Ecken von jedem Frustum um Shadow Map Frustum zu definieren
TODO



\subsection{Implementierung}

% Rendern der Shadow Map
Die Shadow Maps für die Kaskaden werden mittels Texture Arrays realisiert, wobei jede Kaskade einen eigenen Layer in dem Array erhält.
Beim Rendern der Shadow Map transformiert der Vertex Shader die Geometrie zunächst nur in das Welt Koordinatensystem und nicht direkt in das Frustum der Shadow Map.
Der Geometry Shader klont dann die Geometrie und transformiert sie für jede Kaskade mit der dazugehörigen Projektionsmatrix in das Shadow Map Frustum um sie in den entsprechenden Layer der Shadow Map zu rendern.
Anstatt die Near Plane des Shadow Map Frustums zu verschieben, um Geometrie die aus der Sicht des Lichtes vor dem Kamera Frustum liegt zu berücksichtigen, wird an der Near Plane des Shadow Map Frustums der Tiefenwert geclampt.

% Reduktion 1


% Temporäre Licht Projektion


% Reduktion 2

% Keks


\paragraph{Erster Versuch}
Im ersten Ansatz haben wir die Reduktionen für die tatsächliche Near und Far Distanz sowie die Bounding Boxen der Geometrie in den Kaskaden auf der GPU berechnet.
Das Ergebnis haben wir auf die CPU zurück gelesen um aus den Near und Far Distanzen die Partition zu berechnen, bzw. aus den Bounding Boxen das Frustum für die entsprechende Shadow Map.
Um zu vermeiden, dass der CPU auf die GPU warten muss, haben wir die Ergebnisse aus dem vorherigen Frame, bzw. von vor zwei Frames genutzt, da deren Berechnung dann schon abgeschlossen sein sollte.
Um Bewegungen der Kamera zu kompensieren, haben wir die Ergebnisse aus dem Kamera Koordinatensystem des vergangenen Frames in die Welt transformiert und von dort in das aktuelle Kamera Koordinatensystem.
Was wir nicht kompensieren konnten, war das erscheinen neuer Geometrie, die durch die Bewegung der Kamera sichtbar wurde.
Das hat z.B. zu Bereichen zwischen den Kaskaden geführt, die nicht von einer Shadow Map abgedeckt wurden.

\paragraph{Zweiter Versuch}
Dieses Problem haben wir gelöst indem wir am Anfang jedes Frames einmal nur die Tiefe gerendert haben.
Anschließend haben wir darauf die Near und Far Distanz berechnet.
Anstatt auf die CPU zurück zu lesen, haben wir die Partition in einem Compute Shader auf der GPU berechnet.
Nachdem dann die Bounding Boxen der Kaskaden berechnet waren, haben wir daraus ebenfalls auf der GPU die Projektionsmatrizen für die Shadow Maps berechnet.

\section{Volumetric Obscurance}

	TODO
	
\section{Moment Shadow Mapping}
Um eine möglichst hohe Qualität an Schatten zu erreichen bieten sich verschiedene Verfahren an. Deswegen ist es wünschenswert für uns eine filterbare Daten zu nutzen. Percentage Close Filtering eliminiert effektiv Aliasing Effekt durch samplen der Schadow Map durch einen Filter Kernel.\cite{msm} Allerdings kann dies nur für einzelne Fragmente möglich, da das Verfahren Tiefenabhängig beim Schattentest ist.\citep{msm}
Es bietet sich daher an mit Momenten zu arbeiten. Variance Shadow Maps nutzen zwei Momente und bietet deshalb zusätzlich die quadrierte Tiefeninformation. Diese wird genutzt um PCF weiter zu approximieren.



\subsection{Algorithmus}

\subsection{Implementierung}
Wir implementieren die Hamburger Moment Shadow Mapping Technik, da diese die schnellste und robusteste Methode für das Shadow Mapping darstellt.\cite{msm} Zuerst wird eine Moment Shadow Map gerendert. Der Unterschied zur normalen Shadow Map ist, das zusätzlich noch 4 Momente, bestehend aus der Tiefe der Shadow Map gespeichert werden. Diese reduntanten Tiefeninformationen können noch gefiltert werden. Wir verwenden hierfür einen Gauss Filter oder alternativ noch MSAA. Eine weitere Möglichkeit wäre Mip-Mapping.
Im zweiten Schritt der Implementierung, wird ein gefiltertes Sample aus der Moment Shadow Map, sowie die Tiefe eines Shadow Map Fragments genutzt, um eine gefilterte Schatten Intensität zu erhalten. Dies wird durch den Algorithmus zwei aus dem Moment Shadow Mapping Paper erreicht.\cite{msm}
Dadurch erhalten wir ein Intensität für dieses Fragment.

Für das Moment Based Volumetric Obscurance benutzen wir die gleiche Herangehensweise. Allerdings hat es sich herausgestellt, dass die Transformation für 4 Momente zu Artefakten führt, da die numerische optimierung mit einer der Matritzen, wie im MSM Paper beschrieben zu Artefakten führt. Problematisch ist hier die Varianz um einen möglichst korrekten Wert für die Vomuletric Obscurance zu erhalten.
Wir wählen ebenfalls 4 Momente um die Volumetric Obscurance zu approximieren und berechnen die benötigten Gewichte durch eine Abbänderung der MSM Funktionen.



	


\section{Fazit}

	TODO

% Literaturverzeichnis:
\bibliography{lit}
\bibliographystyle{alpha}

\end{document}

